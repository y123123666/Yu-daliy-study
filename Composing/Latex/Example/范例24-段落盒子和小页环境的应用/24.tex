% /*
%  * @Author: yu 
%  * @Date: 2024-10-25 21:50:14 
%  * @Last Modified by:   yu 
%  * @Last Modified time: 2024-10-25 21:50:14 
%  */
\documentclass{ctexart}
\usepackage[paperheight=18cm,paperwidth=18cm,bottom=2mm,left=2mm,right=2mm,top=2mm]{geometry}% 纸张大小
\usepackage{paralist,graphicx,xcolor}% 列表,图片,颜色
\usepackage{amsmath}% 公式
\usepackage{dashbox}% 虚线框命令\dbox
\usepackage{lmodern}% 使用可缩放字体:使用可缩放的字体包,例如 lmodern 或 newtxmath,这些字体包支持更多的字体大小。
\pagestyle{empty}% 无页码

\begin{document}

\makebox[0.3em][r]{(\hspace{-0.15em}}1\makebox[0.4em][l]{)\hspace{-0.15em}}字符与段落盒子混排\\
% 正弦定理:\parbox[t]{4cm}{三角形的边长与改边所对角的正弦值的比值为一个常数。}

% 正弦定理:\parbox[b]{4cm}{三角形的边长与改边所对角的正弦值的比值为一个常数。}

% 正弦定理:\parbox[c]{4cm}{三角形的边长与改边所对角的正弦值的比值为一个常数。}

正弦定理:\parbox[t][2.5cm][b]{4cm}{三角形的边长与改边所对角的正弦值的比值为一个常数。}

\makebox[0.3em][r]{(\hspace{-0.15em}}2\makebox[0.4em][l]{)\hspace{-0.15em}}字符与小页环境混排\\

正弦定理:\begin{minipage}[t]{4cm}三角形的边长与改边所对角的正弦值的比值为一个常数。
\end{minipage} 还有余弦定理。

\makebox[0.3em][r]{(\hspace{-0.15em}}3\makebox[0.4em][l]{)\hspace{-0.1em}}含有多个段落盒子\makebox[0.3em][r]{(\hspace{-0.15em}}或小页环境\makebox[0.4em][l]{)\hspace{-0.15em}}的混排\\

正弦定理:\parbox[t]{4cm}{三角形的边长与改边所对角的正弦值的比值为一个常数。} \qquad%
余弦定理:\parbox[6]{6.2cm}{三角形的两条边长的平方和与第三边的平方差再除以前面两条边积的两倍等于第三边所对角的余弦值。}

\end{document}