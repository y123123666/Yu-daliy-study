% /*
%  * @Author: yu 
%  * @Date: 2024-10-25 21:50:14 
%  * @Last Modified by:   yu 
%  * @Last Modified time: 2024-10-25 21:50:14 
%  */
\documentclass{ctexart}
\usepackage[paperheight=18cm,paperwidth=13cm,bottom=2mm,left=2mm,right=2mm,top=2mm]{geometry}% 纸张大小
\usepackage{paralist,graphicx,xcolor}% 列表,图片,颜色
\usepackage{amsmath}% 公式
\usepackage{dashbox}% 虚线框命令\dbox
\usepackage{lmodern}% 使用可缩放字体:使用可缩放的字体包,例如 lmodern 或 newtxmath,这些字体包支持更多的字体大小。

% \setlength{\textwidth}{<1000>}% 设置文本宽度
\pagestyle{empty}% 无页码

\begin{document}

学习\LaTeX 要有数学头脑。要认真理解
\begin{minipage}{1.76cm}
\fbox{\parbox[t]{1.57cm}{段落盒子\\小页环境}}
\end{minipage}
才能利用好他们。

\rule[-1cm]{0mm}{0mm}

学习\LaTeX 要有数学头脑。要认真理解
\begin{minipage}[t]{1.76cm}
\fbox{\parbox[t]{1.57cm}{段落盒子\\小页环境}}
\end{minipage}
才能利用好他们。

\rule[-1cm]{0mm}{0mm}

学习\LaTeX 要有数学头脑。要认真理解
\begin{minipage}[b]{1.76cm}
\fbox{\parbox[t]{1.57cm}{段落盒子\\小页环境}}
\end{minipage}
才能利用好他们。

\rule[-1cm]{0mm}{0mm}

学习\LaTeX 要有数学头脑。要认真理解
\begin{minipage}[b]{1.76cm}
\hrule height 0pt
\fbox{\parbox[t]{1.57cm}{段落盒子\\小页环境}}
\end{minipage}
才能利用好他们。

\rule[-1cm]{0mm}{0mm}

学习\LaTeX 要有数学头脑。要认真理解
\begin{minipage}[b]{1.76cm}
\fbox{\parbox[t]{1.57cm}{段落盒子\\小页环境}}
\hrule height 0pt
\end{minipage}
才能利用好他们。

\end{document}