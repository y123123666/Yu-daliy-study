\documentclass[a4paper,12pt]{ctexart}
\usepackage{amsmath, amssymb, amsthm}  % 数学公式
\usepackage{graphicx}                  % 插入图片
\usepackage{hyperref}                  % 超链接
\usepackage{listings}                  % 插入代码
\usepackage{color}                     % 颜色
\usepackage{geometry}                  % 页面布局
\geometry{left=3cm,right=3cm,top=2.5cm,bottom=2.5cm} % 页面边距

% 定义代码样式
\lstset{
    language=Python,                    % 语言选择
    basicstyle=\ttfamily,               % 代码字体
    keywordstyle=\color{blue},          % 关键词颜色
    commentstyle=\color{green},         % 注释颜色
    stringstyle=\color{red},            % 字符串颜色
    frame=single,                       % 代码框
    breaklines=true,                    % 自动换行
    columns=fullflexible
}

% 定义标题和作者信息
\title{Learning Notes}
\author{小袁}
\date{\today}

\begin{document}

\maketitle

\tableofcontents   % 自动生成目录
\newpage

% 第一章:引言
\section{Introduction}
这部分可以简要介绍你当前的学习内容或目标。

\subsection{Purpose}
解释本次笔记的目的及重点内容。

% 数学部分
\section{Mathematics Notes}
这里可以记录一些数学公式和推导过程。

例如:二次方程的求解公式
\begin{equation}
x = \frac{-b \pm \sqrt{b^2 - 4ac}}{2a}
\end{equation}

% 图片部分
\section{Graphics}
插入图片示例:
\begin{figure}[h]
\centering
\includegraphics[width=0.5\textwidth]{example-image} % 这里替换为你的图片路径
\caption{Sample Image}
\end{figure}

% 代码部分
\section{Code Example}
插入代码片段:
\begin{lstlisting}
def fibonacci(n):
    a, b = 0, 1
    for _ in range(n):
        print(a)
        a, b = b, a + b
\end{lstlisting}

% 其他内容
\section{Summary and Thoughts}
这一部分可以用于总结学习的要点或记录你的反思。

\subsection{Key Takeaways}
列出本次学习中重要的知识点。

\end{document}
